\let\negmedspace\undefined
\let\negthickspace\undefined
\documentclass{article}
\usepackage{cite}
\usepackage{amsmath,amssymb,amsfonts,amsthm}
\usepackage{algorithmic}
\usepackage{graphicx}
\usepackage{textcomp}
\usepackage{xcolor}
\usepackage{txfonts}
\usepackage{float}
\usepackage{listings}
\usepackage{enumitem}
\usepackage{mathtools}
\usepackage{gensymb}
\usepackage{tfrupee}
\usepackage[breaklinks=true]{hyperref}
\usepackage{tkz-euclide} % loads  TikZ and tkz-base
\usepackage{listings}
\usepackage{gvv}

%usetkzobj{all}
%    \usepackage{color}                                            %%
%    \usepackage{array}                                            %%
%    \usepackage{longtable}                                        %%
%    \usepackage{calc}                                             %%
%    \usepackage{multirow}                                         %%
%    \usepackage{hhline}                                           %%
%    \usepackage{ifthen}                                           %%
  %optionally (for landscape tables embedded in another document): %%
%    \usepackage{lscape}
%\usepackage{multicol}
%\usepackage{chngcntr}
%\usepackage{enumerate}

%\usepackage{wasysym}
%\documentclass[conference]{IEEEtran}
%\IEEEoverridecommandlockouts
% The preceding line is only needed to identify funding in the first footnote. If that is unneeded, please comment it out.

\newtheorem{theorem}{Theorem}[section]
\newtheorem{problem}{Problem}
\newtheorem{proposition}{Proposition}[section]
\newtheorem{lemma}{Lemma}[section]
\newtheorem{corollary}[theorem]{Corollary}
\newtheorem{example}{Example}[section]
\newtheorem{definition}[problem]{Definition}
%\newtheorem{thm}{Theorem}[section]
%\newtheorem{defn}[thm]{Definition}
%\newtheorem{algorithm}{Algorithm}[section]
%\newtheorem{cor}{Corollary}
\newcommand{\BEQA}{\begin{eqnarray}}
\newcommand{\EEQA}{\end{eqnarray}}
%\newcommand{\define}{\stackrel{\triangle}{=}}
\theoremstyle{remark}
\newtheorem{rem}{Remark}

%\bibliographystyle{ieeetr}
%\begin{document}


\def\mytitle{Platformio ASSIGNMENT }
\def\myauthor{Vooribindi Chandini}
\def\mycontact{chandini.vooribindi@gmail.com}
\def\mymodule{IITH-Future Wireless Communication}

%\documentclass[journal,12pt,twocolumn]{IEEEtran}
\usepackage{graphicx} 
\usepackage{enumitem}
\usepackage{tikz}
\usepackage{circuitikz}
\usepackage{karnaugh-map}
\usepackage{tabularx}
\title{\mytitle}
\author{\myauthor\\\mycontact\\IITH\hspace{0.3em}-\hspace{0.3em}\mymodule}
\begin{document}
\maketitle

\hfill(GATE EE 2023)
\begin{enumerate}
    \item  Neglecting the delays due to the logic gates in the circuit shown in figure, the 
decimal equivalent of the binary sequence [ABCD] of initial logic states, which 
will not change with clock, is 

\end{enumerate}
 \begin{figure}[H]
        \centering
        \includegraphics[width = \columnwidth]{figs/ec2023.jpeg}
        \caption{D-Flip-flop}
        \label{ec2023.jpeg}
    \end{figure}  
    
\begin{center}
Fig. 1 
\end{center}

\end{document}
